\documentclass[a4paper,12pt]{article}
\usepackage[mathscr]{euscript}

\title{StorageCoin: A cryptocurrency designed for a storage system}

\author{Jean-Louis Gu\'{e}n\'{e}go, Yannis Thomias}
\date{}

\bibliographystyle{plain}

\begin{document}
\maketitle
\tableofcontents

\section{Introduction}
~\cite{bitcoin}
~\cite{peercoin}
~\cite{smallcell}

\section{Model description}
\subsection{Provider}

\paragraph*{}
The system is a distributed network with $N$ nodes that belong to a set $\mathscr{P}$ of \emph{providers} $\mathscr{P}_{i}$.
Each node has the same storage capacity $v$.
$N$ can evolve during time because providers can add or remove nodes  when desired. Each provider $\mathscr{P}_{i}$ has $N_{i}$ nodes.
$N$ and $N_{i}$ are function of time. We have 
\[N=\sum\limits_{i\in{\mathscr{P}}}N_{i}\]

\paragraph*{}
The variables $N$, $N_{i}$ depend of time, but we just need representing them according the cycle. We note the cycle $n$.
So we have $N=N(n)$, $N_{i}=N_{i}(n)$. 

\paragraph*{}
The StorageCoin (STC) has a notion of \emph{cycle}.
Every cycle the system injects $G(n)$ STC shared between the providers.
Each provider $\mathscr{P}_{i}$ receives a \emph{gain} $G_{i}$ STC such as
\[G_{i}(n)=\frac{N_{i}(n)}{N(n)}G(n)\]

\paragraph*{}
We note $T(n)$ the number of STC injected into the system. We have
\[T(n)=\sum\limits_{t=1}^{n}G(t)\]

\paragraph*{}
For each provider, we note $T_{i}$ the current amount of STC received from the beginning. We have
\[T_{i}(n)=\sum\limits_{t=1}^n G_{i}(t)\]

\paragraph*{}
For each actor we note $U_{i}$ the current amount of STC owned by this actor. We have foe each provider: 
\[U_{i}(n)\le{T_{i}(n)}\]
We have the equality if the provider has never sold any STC.

\paragraph*{}
Every cycle the system applies a flat tax $F_{i}(n)=f(n)U_{i}(n)$ to every actor, both providers and consumers. The purpose of this tax is to regulate $T(n)$. 
We want that $T(n)$ does not exceed a maximum value $T_{max}$. 
For that we define $f(n)=0$ when $T<T_{max}$ and otherwise \[f(n)=\frac{G(n)}{T(n)+G(n)}\]

We are going to study two cases. 
A first case which we assume that $G(n)$ is constant: $G(n)=G_{0}$. 
And a second case where providers are rewarded proportionally to there number of nodes: $G_{i}(n)=gN_{i}(n)$. 
$g$ is a number of STC that a provider receives for one node.

\paragraph*{}
The provider has internal cost price to maintain his nodes operational on the system. We assume that this cost is proportional to the storage memory exposed. We note $c_{i}$ the cost price per memory unit for the provider $\mathscr{P}_{i}$. The provider cost price is
\[C_{i}(n)=N_{i}(n)vc_{i}\]

\paragraph*{}
The StorageCoin can be exchanged on a free market, through dedicated STC marketplaces. We note $p$ the exchange rate for 1 STC in a fiat money. We use the dollar unit (\$) in this document.
In this model we assume that $p$ varies only over cycles. We note $p(n)$.
$p(n)$ increases or decreases according to the quantity of offer and demand as all free market process.

\paragraph*{}
The provider \emph{activity revenue} $R_{i}$ is the conversion in the fiat money of $G_{i}(n)$. Therefore we have 
\[R_{i}(n)=G_{i}(n)p(n)\]

\paragraph*{}
We can notice that we don't include the flat tax in the provider activity revenue. Indeed the flat tax does not depend of the storage contribution activity, but on the STC current amount of the provider.

\paragraph*{}
A provider will be interested to expose nodes only if 
$R_{i}(n)\ge{C_{i}}$.

\subsection{Consumer}
\paragraph*{}
\emph{Consumers} are users of the system that buy with a fiat money some STC to use the storage resources of the system for storing objects (files, etc.). 
Buying a percentage $e$ of all the STC of the system $T$ means being allowed to use the percentage $e$ of all the current storage resources $V(n)=N(n).v$.

\paragraph*{}
Consumers will be interested to use the system if it presents advantages regards to the competition.
At this time competition pricing model is similar to a rental business model which is different of the StorageCoin model. 
In order to compare both models regarding prices, we consider an example.

\paragraph*{}
In the example a consumer buys at cycle $n$ a volume $A$ of storage. $A$ is measured in a storage unit like the Gigabyte. After $m$ cycles, the consumer decide to resell its storage resources by selling them.

\paragraph*{}
For the competition, if we consider that there is no minimal commitment period, that the customer can cancel when desired, and the cost per cycle is $d$, then the cost $D_{c}$ will be easy to compute: $D_{c}=m.A.d$

\paragraph*{}
For the StorageCoin system, the computation is made differently. In order to buy storage resources, the consumer must buy enough STC from the providers at the price $p(n)$ of the market at cycle $n_{0}$.
To get $A$, the consumer $\mathscr{C}_{i}$ will have to buy the amount $S_{A}$ of STC given by 
\[S_{A}=\frac{T(n)}{V(n)}A\]
Let $S_{i}(n)$ be the amount of STC owned by the consumer $\mathscr{C}_{i}$.
We have \[S_{i}(n_{0})=S_{A}\]
At the next cycle, $G$ STC will be injected into the system if $T(n)<T_{max}$. 
Otherwise, a flat tax $f(n)$ is applied to the consumer and $T(n)$ is constant.
Maybe the number of nodes in the system will be different (bigger or smaller). 
Therefore the amount $S_{A}$ will not reflect anymore the volume $A$ of storage resources 
\[S_{A}\neq \frac{T(n_{0}+1)}{V(n_{0}+1)}A\]
Furthermore, the amount $S_{i}$ is decreasing when the tax applies:
\[S_{i}(n+1)=S_{i}(n)(1-f(n))\]

\paragraph*{}
The amount of space $A(k)$ owned by the consumer $\mathscr{C}_{i}$, where $k$ is the number of cycles spent after $n_{0}$, corresponding to $S_{i}$ will be: 
\[A(k)=\frac{V(n_{0}+k)}{T(n_{0}+k)}S_{i}(n_{0}+k)\]
$A(k)$ can be smaller or bigger than $S_{A}$. If the number of nodes $N$ in the system is constant across the cycles, then we have $V(n+k)=V(n)$, and because $T(n+k)=T(n)+kG$, we have
\[A(k)=\frac{V(n)}{T(n)+kG}S_{A}\le A(0)=A\]
In this case the consumer would have to buy more STC in order to dispose of volume $A$ in the system.
However if $n$ is big enough, the amount of STC to buy would be small.
Instead of buying a small amount at every cycle to maintain the volume $A$ of storage, the consumer should better buy a little bit more of what he needs at the cycle $n$.
For more simplicity, we assume that the consumer does not need to buy extra space during the $m$ cycles.
After $m$ cycles, he can resell his STC. The STC will be sold at the current price of the market $p(n + m)$.
The real cost of renting the volume $A$ on the StorageCoin system is therfore
\[D_{s}=S_{A}(p(n)-p(n+m))\]

\paragraph*{}
The consumer will be interested to use the StorageCoin system if it is cheaper than the competition, in other word if we have in most cases $D_{s}\le D_{c}$.
If the STC exchange rate increases, the consumer will even make profit. 
This business model is similar to buy a house, rent or use it, then sell it.
If the STC exchange rate decreases, the consumer will check that the decreasing is not too much important in order to still have 
$D_{s}\le D_{c}$. In other words, with $b$ the cycle when the storage is bought, and $s$ the cycle when the storage is sold:
\[\frac{T(b)(p(b)-p(s))}{V(b)d(s-b)}\le 1\]
Before to buy storage resources on the StorageCoin system, the consumer will look at the past evolution of the exchange rate of STC.
In average on the duration of renting memory $s-b$, the condition becomes for the cycle $n$:
\[\frac{2}{n(n-1)}\sum\limits_{b=1}^{n-1} \sum\limits_{s=b+1}^{n}\frac{T(b)(p(b)-p(s))}{V(b)d(s-b)}\le 1\]

\bibliography{ocpbiblio}
\end{document}
