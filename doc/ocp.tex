\documentclass[a4paper,12pt]{article}

\title{StorageCoin: A cryptocurrency designed for a storage system}

\author{Jean-Louis Gu\'{e}n\'{e}go, Yannis Thomias}
\date{}

\bibliographystyle{plain}

\begin{document}
\maketitle \clearpage
\tableofcontents \clearpage

\section{Introduction}
~\cite{bitcoin}
~\cite{peercoin}
~\cite{smallcell}

\section{Model description}
\subsection{Provider}

\paragraph*{}
The system is a distributed network with $N$ nodes that belong to a set $P$ of \emph{providers} $P_{i}$.
Each node has the same storage capacity $v$.
$N$ can evolve during time because providers can add or remove nodes  when desired. Each provider $P_{i}$ has $N_{i}$ nodes.
$N$ and $N_{i}$ are function of time. We have 
\[N=\sum\limits_{i\in{P}}N_{i}\]

\paragraph*{}
The variables $N$, $N_{i}$ depend of time, but we just need representing them according the cycle. We note the cycle $n$.
So we have $N=N(n)$, $N_{i}=N_{i}(n)$. 

\paragraph*{}
The StorageCoin (STC) has a notion of \emph{cycle}.
Every cycle the system injects $G$ STC shared between the providers.
Each participant $P_{i}$ receives a \emph{gain} $G_{i}$ STC such as
\[G_{i}(n)=\frac{N_{i}(n)}{N(n)}.G\]

\paragraph*{}
We note $T(n)$ the number of STC injected into the system. We have $T(n)=nG$.

\paragraph*{}
For each provider, we note $T_{i}$ the current amount of STC received from the beginning. We have
\[T_{i}(n)=\sum\limits_{t=1}^n G_{i}(t)\]

\paragraph*{}
The provider have internal cost price to maintain his nodes operational on the system. We assume that this cost is proportional to the storage memory exposed. We note $c_{i}$ the cost price per memory unit for the provider $P_{i}$. The provider cost price is
\[C_{i}=N_{i}.v.c_{i}\]

\paragraph*{}
The StorageCoin can be exchange on a free market, through dedicated STC marketplaces. We note $p$ the exchange rate for 1 STC in a fiat money. We use the dollar unit (\$) in this document.
In this model we assume that $p$ varies only over cycles. We note $p(n)$.
$p(n)$ increases or decreases according to the quantity of offer and demand as all free market process.

\paragraph*{}
The provider revenue $R_{i}$ is the conversion of its gain $G_{i}$ in the fiat money. Therefore we have 
\[R_{i}(n)=G_{i}(n).p(n)\]

\paragraph*{}
A provider will be interested to expose nodes only if $R_{i}(n)\ge{C_{i}}$.

\paragraph*{}
\emph{Consumers} are users of the system that buy with a fiat money some STC to use the storage resources of the system for storing objects (files, etc.). Buying a percentage $F$ of all the STC of the system $T$ means being allowed to use the percentage $F$ of all the current storage resources $V(n)=N(n).v$.

\subsection{Consumer}
\paragraph*{}
Consumers will be interested to use the system if it presents advantages regards to the competition.
At this time competition pricing model is similar to a rental business model which is different of the StorageCoin model. 
In order to compare both models regarding prices, we consider an example.

\paragraph*{}
In the example a consumer buys at cycle $n$ a volume $A$ of storage. $A$ is mesured in a storage unit like the Gigabyte. After $M$ cycles, the consumer decide to retaliate it storage resources by selling them.

\paragraph*{}
For the competition, if we consider that there is no minimal commitment period, that the customer can cancel when desired, and the cost per cycle is $d$, then the cost $D_{c}$ will be easy to compute: $D_{c}=M.A.d$

\paragraph*{}
For the StorageCoin system, the computation is made differently. In order to buy storage resources, the consumer must buy enough STC from the providers at the price $p(n)$ of the market of cycle $n$.
To get $A$ the amount $S_{A}$ of STC to buy is given buy 
\[S_{A}=\frac{T(n)}{V(n)}.A\]
At the next cycle, $R$ STC will be injected into the system. Maybe the number of node in the system will be different (bigger or smaller). 
Therefore the amount $S_{A}$ will not reflect anymore the volume $A$ of storage resources 
\[S_{A}\neq \frac{T(n+1)}{V(n+1)}.A\]
The amount of space $A(k)$, where $k$ is the number of cycles spent after $n$, corresponding to $S_{A}$ will be: 
\[A(k)=\frac{V(n+k)}{T(n+k)}.S_{A}\]
$A(k)$ can be smaller or bigger than $S_{A}$. If the number of nodes $N$ in the system is constant across the cycles, then we have $V(n+k)=V(n)$, and because $T(n+k)=T(n)+kR$, we have
\[A(k)=\frac{V(n)}{T(n)+kR}.S_{A}\le A(0)=A\]
In this case the consumer would have to buy more STC in order to dispose of volume $A$ in the system.
However if $n$ is big enough, the amount of STC to buy would be small.
Instead of buying a small amount at every cycle to maintain the volume $A$ of storage, the consumer should better buy a little bit more of what he needs at the cycle $n$.
For more simplicity, we assume that the consumer does not need to buy extra space during the $M$ cycles.
After $M$ cycles, he can resell his STC. The STC will be sold at the current price of the market $p(n + M)$.
The real cost of renting the volume $A$ on the StorageCoin system is therfore
\[D_{s}=S_{A}(p(n)-p(n+M))\]
The consumer will be interested to use the StorageCoin system if it is cheaper than the competition, in other word if we have in most cases $D_{s}<D_{c}$.
If the STC exchange rate increases, the consumer will even make profit. 
This business model is similar to buy a house, rent or use it, then sell it.
If the STC exchange rate decreases, the consumer will check that the decreasing is not too much important in order to still have $D_{s}<D_{c}$. 

\bibliography{ocpbiblio}
\end{document}
